% !TEX root = thesis-main.tex

\begin{savequote}[45mm]
--- Il n'y a pas de hors-texte.
\qauthor{Jacques Derrida}
\end{savequote}

\chapter{Introduction}


\label{chapter:introduction}

%%%%%%%%%%%%%%%%%%%%%%%%%%%%%%%%%%%%%%%%%%%%%%%%%%%%%%
%  OPENING
%%%%%%%%%%%%%%%%%%%%%%%%%%%%%%%%%%%%%%%%%%%%%%%%%%%%%%

% NN learning from data
Neural networks learn patterns from data to solve complex problems.
To understand and infer meaning in language, neural models have to learn complicated nuances.

% detour to nuances of language
Discovering distinctive linguistic phenomena from data is not an easy task. 
For instance, lexical ambiguity is a fundamental feature of language which is challenging to learn \citep{small2013lexical}. 
Even more prominently, inferring the meaning of rare and unseen lexical units is difficult with neural networks \citep{koehn2017six}. 
For instance, \citet{rios-etal-2018-word} provide an example where an English-German translation model translates the sentence ``[$\ldots$] \textit{Hedge-Fund- \textbf{Anlagen} nicht zwangsl\"{a}ufig risikoreicher sind als traditionelle  \textbf{Anlagen}}'' to 
``[$\ldots$] \textit{hedge fund \underline{assets} are not necessarily more risky than traditional \underline{plants}}''. Here, the ambiguous word \textit{`Anlagen'} is first translated correctly to \textit{`assets'}, but then incorrectly to \textit{`plants'} in the second occurrence.

To understand many of these phenomena, a model has to learn from a few instances and be able to generalize well to unseen cases.
%
% connect using context to learn nuances of language
%
Natural language speakers typically learn the meanings of words by the \textit{context} in which they are used.
 \citet{miller-1985-dictionaries} states that:
\begin{quote}
``\emph{When subtle semantic distinctions are at issue, it is customary to remark that a satisfactory language understanding system will have to know a great deal more than the linguistic values of words.}''
\end{quote}

\noindent Sentence and document-level context provide the possibility to go beyond lexical instances and study words in a broader context.
Neural models use a sizable amount of data that often consists of contextual instances to learn patterns.
% back to learning from data and the necessity of {data-driven} learning 
However, the learning process is hindered when training data is scarce for a task \citep{45801,edunov-etal-2018-understanding}.
Even with sufficient data, learning patterns for the long tail of the lexical distribution is challenging \citep{NIPS2017_7278}. 
To address these problems, one approach is to augment the training data \citep{sennrich-haddow-birch:2016:P16-11}.
Many strategies for data augmentation focus on increasing the amount of data to assist the learning process of data-driven neural models.
While simply increasing the size of data is helpful, it is not entirely clear \textit{where} the improvements come from and \textit{how} neural models benefit from the additional context with augmentation. 

%%%%%%%%%%%%%%%%%%%%%%%%%%%%%%%%%%%%%%%%%%%%%%%%%%%%%%
%  FIRST PART OF THE TITLE: "ENHANCING THE USE OF CONTEXT [...]"
%%%%%%%%%%%%%%%%%%%%%%%%%%%%%%%%%%%%%%%%%%%%%%%%%%%%%%
% What is important about context and why we want to understand and enhance it?
Arguably, it is important to understand the impact of new contexts to design augmentation models that exploit these contexts.
This includes understanding what constitutes a beneficial context, and how to enhance the use of context in neural models.
In this thesis, we focus on understanding certain potentials of contexts in a neural model, and design augmentation models to benefit from them. 
%We are particularly interested in the use of context in understanding meaning in a language.  

%%%%%%%%%%%%%%%%%%%%%%%%%%%%%%%%%%%%%%%%%%%%%%%%%%%%%%
%  SECOND PART OF THE TITLE: "[...] FOR MACHINE TRANSLATION"
%%%%%%%%%%%%%%%%%%%%%%%%%%%%%%%%%%%%%%%%%%%%%%%%%%%%%%
% why machine translation?
We focus on machine translation as a prominent instance of the more general language understanding problems. 
In order to translate from a source language to a target language, a neural model has to understand the meaning of constituents in the provided context and generate constituents with the same meanings in the target language.
This task accentuates the value of capturing nuances of language and the necessity of generalization from few observations \citep{DBLP:journals/corr/abs-2004-02181}.
Additionally, the lack of large amounts of labeled data is even more pronounced in machine translation in the form of bilingual corpora.
This signifies the need for efficient and informed data augmentation models.

%%%%%%%%%%%%%%%%%%%%%%%%%%%%%%%%%%%%%%%%%%%%%%%%%%%%%%
%  CLOSING THE INTRO SECTION IF THE INTRODUCTION CHAPTER!
%%%%%%%%%%%%%%%%%%%%%%%%%%%%%%%%%%%%%%%%%%%%%%%%%%%%%%
\medskip

% focus on _where_ and _why_
The main problem we study in this thesis is what neural machine translation models (NMT) learn from data, and how we can devise more focused contexts to enhance this learning.
We believe that looking more in-depth into the role of context and the impact of data on learning models is essential to advance the Natural Language Processing (NLP) field. 
Understanding the importance of data in the learning process and how neural network models interact, utilize, and benefit from data can help develop more {accurate} NLP systems. 
Moreover, it helps highlight vulnerabilities and volatilities of current neural networks and provides insights into designing more  robust models.




%% The research questions and sub questions
% !TEX root = thesis-main.tex

\section{Research outline and questions}
\label{section:introduction:rqs}


This thesis explores the role of context in language understanding and in particular, machine translation using recent advances in deep learning.
We develop novel models and learning algorithms to examine the abilities of neural networks in learning from data. 
Specifically, we are interested in the importance of contextual cues in translating words and various ways we can use data to advance translation systems further. 

Before investigating the role of context in the bilingual setting of machine translation, we ask ourselves a more general question about the impact of context in monolingual settings.
As a preliminary investigation into this question, we look into ambiguous words where, by definition, context is the prominent factor in understanding word meaning. 
We study how document-level contexts as topics aid in distinguishing different meanings of a word. 

Next, in more detail, we focus on the influence of context in the bilingual setting of machine translation. 
While recent advances in neural networks have been very successful in translation, the significance of different aspects of data is still largely unexplored.
We investigate how the translation models exploit context to learn and transfer meaning and show that manipulating data improves translation quality.
In particular, our proposed models examine how different and diverse contexts resolve various obstacles of translation.

Lastly, we address the shortfalls of relying only on the observed context to learn word meaning and focus on particularly interesting cases. 
Neural networks optimize the learning process on the available data. 
We examine under which conditions the observed context in the training data is not enough for meaning inference and capturing various linguistic phenomena. 
Moreover, we raise questions about the learning abilities of current translation models and where they fail to capture the available information in data. 
With contextual modifications, we identify an underlying generalization problem in state-of-the-art translation models.

Concretely, we set out to answer the following research questions in this thesis:


\acrodef{rq:topic}[\ref{rq:topic}]{\textit{Can document-level topic distribution help infer the meaning of a word?}}
\acrodef{rq:topic1}[\ref{rq:topic1}]{\textit{ To what extent can distributions over word senses be approximated by distributions over topics of documents without assuming these concepts to be identical?}}  % [embedding]}}
\acrodef{rq:topic2}[\ref{rq:topic2}]{\textit{ How can we exploit document-level topics to distinguish between different meanings of a word and learn the corresponding representations?}}%  [embedding]}}
\acrodef{rq:topic3}[\ref{rq:topic3}]{\textit{ What are the advantages of using document-level topics in learning multiple representations per word? }}% [embedding]}}

\acrodef{rq:tdabt}[\ref{rq:tdabt}]{\textit{How is the translation quality of a word influenced by the availability of diverse contexts?}}

\acrodef{rq:tda1}[\ref{rq:tda1}]{\textit{How can we successfully augment the training data with diverse contexts for rare words?}}%   [augmentation]}}
\acrodef{rq:tda2}[\ref{rq:tda2}]{\textit{Do rare words benefit from augmentation via paraphrasing during test time?}}% [augmentation]}}
\acrodef{rq:bt1}[\ref{rq:bt1}]{\textit{ Do signals from the NMT model help identify low-confidence words that could benefit from additional context?  }}%  [backtrans]}}
\acrodef{rq:bt2}[\ref{rq:bt2}]{\textit{ How can we successfully apply data selection of monolingual data to diversify the contexts of low-confidence words? }}% [backtrans]}}

\acrodef{rq:vol}[\ref{rq:vol}]{\textit{To what extent are neural translation models vulnerable as a result of relying on the observed context in the training data to infer meaning? }}
\acrodef{rq:id1}[\ref{rq:id1}]{\textit{What are the challenges of idiom translation with neural models? }} %   [idiom]}}
\acrodef{rq:id2}[\ref{rq:id2}]{\textit{How is the translation quality of NMT influenced by idiomatic expressions? }} %  [idiom]}}
\acrodef{rq:vol1}[\ref{rq:vol1}]{\textit{ How can contextual modifications during testing reveal a lack of robustness of translation models and affect the translation quality?}}%    [volatility]}}
\acrodef{rq:vol2}[\ref{rq:vol2}]{\textit{ To what extent is a lack of robustness an indicator of a generalization problem in neural machine translation models?  }}% [volatility]}}


\begin{enumerate}[label=\textbf{Research Question \arabic*:},ref={RQ\arabic*},wide = 0pt,resume]
\setlength\itemsep{1em}
\item \acl{rq:topic} \label{rq:topic}

In this research question, we investigate whether using document-level context, as opposed to sentence-level only, has an impact on learning word representations. Word representations are abstract feature vectors that capture word meanings. To produce good representations, the learning model must capture various linguistic phenomena such as the ambiguity of the language.  
Notably, we tackle the problem of representing ambiguous words by defining multiple representations per word and using implicit topics of documents to distinguish between different meanings of a word. 
We divide this research question into three sub-questions and address them in Chapter~\ref{chapter:research-01}:

\begin{enumerate}[label=\textbf{RQ1.\arabic*},wide = 0pt, leftmargin=2em]
\setlength\itemsep{1em}
\item \acl{rq:topic1}  \label{rq:topic1}

\medskip

Modeling document topics is commonly used in different ways to address the challenging task of word sense disambiguation \citep{boyd-graber-etal-2007-topic,li-etal-2010-topic,ChaplotS18}. 
However, the topic of a document does not directly correspond to the senses of the words in that document.
We investigate whether a document topic distribution is an informative signal to help distinguish between different senses of a word and how we can leverage this information to learn word representations. 
Next, we ask:

\item \acl{rq:topic2}  \label{rq:topic2}

\medskip

To answer this question, we estimate document-topic distributions using unsupervised topic modeling techniques. 
We observe that the produced distribution over topics is different for different senses of an ambiguous word. 
We propose three variants of the Skipgram word embedding model \citep{mikolov2013efficient} to integrate topic distributions and learn multiple representations per word. 

\item \acl{rq:topic3}  \label{rq:topic3}

\medskip

To further evaluate our models, we analyze the linguistic phenomena captured by topic-sensitive word representations. 
Namely, we show that different senses of a word are separated into different representations. 
We observe that the additional context of a document topic is most beneficial when the task is more complex.
We find that these representations achieve improvements over the baselines for word similarity and lexical substitution tasks. 

\end{enumerate}

Having examined the effectiveness of learning word representations using auxiliary contextual information, we then investigate how the diversity of the context affects language understanding and transfer of meaning between two languages. Concretely we ask:

\item \acl{rq:tdabt}  \label{rq:tdabt}

In this research question, we choose machine translation as the task of interest.
We investigate this question by diversifying the local context for different words and propose various data augmentation techniques with the new contexts. 
Additionally, we explore the influence of these synthetic contexts on translation quality. 
We divide this research question into four sub-questions and discuss them in Chapters~\ref{chapter:research-02} and~\ref{chapter:research-03} of this thesis. 

\begin{enumerate}[label=\textbf{RQ2.\arabic*},wide = 0pt, leftmargin=2em]
\setlength\itemsep{1em}
\item \acl{rq:tda1} \label{rq:tda1}

\medskip

In this question, we are interested in translation of rare words in low-resource settings where the available data is scarce for one or both languages. 
The success of neural networks is partly due to their ability to learn from vast amounts of data efficiently. 
These models suffer significantly when sufficient data is not available \citep{ngo-etal-2019-overcoming}.
Subsequently, even with adequate data, neural machine translation models have difficulty learning the meaning of rare words existing in the source language \citep{koehn2017six}. 
Additionally, they are also not successful in generating rare words in the target language \citep{luong2014addressing}.  
To answer this question, in Chapter~\ref{chapter:research-02}, we propose a data augmentation technique that targets rare words and substitute them in new sentences with novel contexts. 
Leveraging a monolingual corpus, which is available in much larger quantities in comparison to a bilingual corpus, we create new contexts for rare words in the training data.
We investigate how additional data can improve the learning and the generation of rare words. 
In Chapter~\ref{chapter:research-02}, we show that by increasing the diversity of the contexts of rare words, we can achieve significant improvements in translation quality.

\item \acl{rq:tda2} \label{rq:tda2}

\medskip

Diversifying context is only valid when both source and target sentences are modified, i.e., at training time when the model has access to the sentence pairs.
During inference, only the source sentence is available and we use the reference sentence solely for evaluation. 
As a consequence, any changes to the source sentence have to be meaning-preserving so that we do not modify the reference translations.
We propose a data augmentation technique at test time, focusing on \textit{paraphrasing} rare and unknown words in the source sentence. 
In contrast to our previous approach where the goal was to diversify the context of rare words in the training data, here we substitute rare words with more common synonyms.
In Chapter~\ref{chapter:research-02}, we show that with paraphrasing rare words at test time, we gain improvements in translation quality. 


\item \acl{rq:bt1} \label{rq:bt1}

\medskip

In the previous research questions, we identify rare words as words that can benefit from additional contexts. 
While the translation quality of these words improves with our proposed data augmentation technique, these are not the only words that suffer due to inadequacies in the training data. 
In Chapter~\ref{chapter:research-03}, we expand our investigation in this direction. 
Rather than using features like frequency in the training data, we look into the \textit{model} itself and where it struggles.
We detect the words for which the model has low confidence during translation. 
We examine various approaches to identify these low-confidence words as signaled by the model and augment the training data accordingly. 
Hence, we ask:

\item \acl{rq:bt2} \label{rq:bt2}

\medskip

To generate new contexts and augment the training data, we propose targeted back-translation. 
Back-translation leverages monolingual data in the target language and a trained translation model to translate randomly selected sentences into the source language \citep{sennrich-haddow-birch:2016:P16-11}. 
The automatically generated bilingual data, although noisy, is added to the training data and the translation model is trained on the augmented data.  
In Chapter~\ref{chapter:research-03}, we identify words that can benefit from diverse context.
We show that by back-translating sentences containing low-confidence words, we achieve improvements over the baselines. 

\end{enumerate}

Having demonstrated the advantages of using contextual cues in various forms to improve word representation learning and translation modeling, we come to the final research question of this thesis.
Here, we investigate the shortcomings of relying on the observed context. 
Concretely we ask:

\item \acl{rq:vol} \label{rq:vol}

While the success of neural networks in NLP is indisputable, it is well worth to ask whether neural networks have hidden vulnerabilities. 
In this research question, we also choose machine translation as the task of interest.
We are interested in vulnerabilities of the translation models that can be exposed by looking into the data. 
In particular, we divide this research question into four sub-questions and discuss them in Chapters~\ref{chapter:research-04} and~\ref{chapter:research-05} of this thesis:

\begin{enumerate}[label=\textbf{RQ3.\arabic*},wide = 0pt, leftmargin=2em]
\setlength\itemsep{1em}
\item \acl{rq:id1} \label{rq:id1}

\medskip

Neural translation models struggle in handling idiosyncratic linguistic patterns. 
One of these patterns are idioms, which are semantic lexical units whose meaning is not merely a function of the meaning of its constituent parts.
In Chapter~\ref{chapter:research-04}, we look into idiomatic expressions in particular and why the translation of such phrases is a challenge. 
Furthermore, we automatically label parallel training and test data for idiomatic expressions using a bilingual dictionary of idioms.
We assess whether the sentential context is enough for inferring idiomatic meanings and show that it is indeed not the case.

Next, we ask:

\item \acl{rq:id2} \label{rq:id2}

\medskip

There is no explicit indicator in the data to signal whether a phrase should be translated literally or idiomatically in any given context. 
Researchers have shown that neural models can benefit from side constraints in data in various cases. For instance, 
\citet{sennrich-etal-2016-controlling} note that adding side constraints as unique tokens at the end of the source text help the model translate to the desired level of politeness. 
In Chapter~\ref{chapter:research-04}, we investigate whether a similar technique is useful for the translation of sentences containing idiomatic expressions. 

\medskip

\noindent Finally, we look into other vulnerabilities of neural models which can be highlighted by contextual cues. 
Our next research question focuses on other cases where NMT models fail to generate a correct translation. 
To investigate this question, we first examined how to expose this shortcoming in translation models by asking:  

\item \acl{rq:vol1} \label{rq:vol1}

\medskip

In Chapter~\ref{chapter:research-05}, we ask how receptive the translation models are to manipulations of data. While previous works have investigated the performance of neural models when encountering noise in the form of adversarial instances \citep{goodfellow6572explaining,D18-1050,DBLP:journals/corr/abs-1711-02173}, we are interested in unexpected performance when the data is \textit{not} noisy.

Next, we investigate the robustness of neural translation models by asking:

\item \acl{rq:vol2} \label{rq:vol2}

\medskip

We propose an approach to generate contextual modifications in the test data, yielding semantically and syntactically correct sentences.
Our new test data sheds light on volatile behaviour in current state-of-the-art translation models. 
In Chapter~\ref{chapter:research-05}, we show that identifying this volatility is already achievable with extremely minor modifications.
Our findings highlight unexpected but recurring patterns of errors and possible problems of generalization in neural translation models.

\end{enumerate}
\end{enumerate}



%% Lists the main contributions of the thesis
% !TEX root = thesis-main.tex

\section{Main contributions}
\label{section:introduction:contributions}

Here we summarize the main algorithmic and empirical contributions of this thesis to the field of natural language processing and in particular machine translation, as well as the constructed resources.

\subsection{Algorithmic contributions}

We develop novel learning algorithms and neural network models for investigating the influence of context in learning capacities of models. 

\begin{enumerate}
\item We present a framework for learning multiple embeddings per word using topical context.  With three variants of our model, we employ topical context in various ways and learn distinctions between different senses of the words (Chapter~\ref{chapter:research-01}).
\item We introduce a data augmentation technique for generating new contexts for rare words in machine translation. 
Leveraging monolingual data, we propose a neural language model that given a sentence, suggests rare words to substitute into the given context. 
This new method can be applied to any low-resource language pair as long as there are monolingual data available in both languages (Chapter~\ref{chapter:research-02}).
\item We introduce a novel method to identify difficult words, where the neural translation model has low prediction confidence.
Leveraging this information, we improve upon an existing augmentation technique by replacing its random selection with targeted selection and specifically provide new contexts for low-confidence words (Chapter~\ref{chapter:research-03}).
\item We propose a procedure to (i) automatically detect idiomatic expressions in sentences using a dictionary of idioms, and (ii) automatically annotate the bilingual data with the corresponding idioms (Chapter~\ref{chapter:research-04}).
\item We introduce an effective technique to shed light on the lack of robustness of neural translation models. 
Our approach generates variants of the same sentences that differ slightly and are semantically and syntactically correct. 
We investigate the behaviour of the neural model in translating these variants by proposing metrics to identify volatile performance (Chapter~\ref{chapter:research-05}).
\end{enumerate}

\subsection{Empirical contributions} 

We evaluate our proposed models on large scale data sets as well as controled experiments to validate our hypotheses. We provide empirical results for each research question asked in this thesis. 
More specifically:

\begin{enumerate}
\item We compare how different approaches of incorporating topical context affect the resulting representations. 
We assess the topic-sensitive word representations on word similarity and lexical substitution tasks and perform a qualitative analysis between different representations of a word (Chapter~\ref{chapter:research-01}).
\item We evaluate the effectiveness of our first data augmentation approach in machine translation for two language directions: English$\rightarrow$German and German$\rightarrow$English.
We simulate a low-resource setting by only using a subset of the available training data, while simultaneously being able to compute the upper bound of performance in case more data is available.
Our approach successfully mitigates the problem of rare word translation, where sufficient bilingual training data is not available. 
We perform an analysis of the confidence of the translation model for both generating and translating rare words (Chapter~\ref{chapter:research-02}).
\item We evaluate our second proposed data augmentation approach in machine translation for two language directions: English$\rightarrow$German and German$\rightarrow$English.
We study the effects of previous data augmentation techniques on confidence and the learning capacity of the translation model. 
We compare various ways of identifying low-confidence words and show that targeted data augmentation using these words improves translation quality.
We demonstrate that with diversifying contexts of difficult words, the confidence of the model in predicting these words and consequently the translation quality improve (Chapter~\ref{chapter:research-03}).
\item We conduct an empirical evaluation of translation models facing sentences that include an idiomatic expression. 
Using annotated training and test data, we demonstrate how the current neural translation models struggle with translating idioms.
We show that even when we annotate them in the training data, translating these expressions is a challenge and the translation models require much broader knowledge to learn them (Chapter~\ref{chapter:research-04}).
\item  We show that fluctuations in translations of extremely similar sentences are more prominent than expected. These findings can be used to develop more robust models (Chapter~\ref{chapter:research-05}).
\end{enumerate}



\subsection{Resource contributions}

We release the resources of the proposed models in this thesis including source codes and annotated data. More specifically:

\begin{enumerate}
\item Chapter~\ref{chapter:research-01}: We released the code for the proposed models where we use document topics to learn word representations.
\item Chapter~\ref{chapter:research-02}: We released the code for targeted data augmentation of parallel corpora using language models.
\item Chapter~\ref{chapter:research-04}: We released the annotations of idiomatic phrases in training, development, and test data. The bilingual corpora can be used for translation of English$\rightarrow$German and German$\rightarrow$English.
\item Chapter~\ref{chapter:research-05}: We released a data set which contains multiple variants for each sentence pair in the standard WMT English$\leftrightarrow$German test data.
We annotate the translations of these variants and label different types of errors. Additionally, we release the code for generating sentence variations of bilingual corpora for a more in-depth evaluation of translation quality.
\end{enumerate}







%% Overview of the thesis; what is described in which chapter
% !TEX root = thesis-main.tex

\section{Thesis overview}
\label{section:introduction:overview}

After this introductory chapter, the remainder of this thesis consists of a background chapter (Chapter~\ref{chapter:background}), five research chapters (Chapters~\ref{chapter:research-01}-\ref{chapter:research-05}), and a concluding chapter (Chapter~\ref{chapter:conclusions}). 
Below we present a high-level overview of the main content of each of these chapters. 


\paragraph{Chapter~\ref{chapter:background}: Background} provides an introduction to the neural machine translation (NMT) paradigm used in this thesis. We briefly review the core models, the training and test data required, and the learning and optimization strategies we employ. We also discuss different representation learning approaches. Additionally, we describe the basic experimental settings for our systems. 
Finally, we provide an overview of evaluation metrics used in this thesis.  

\paragraph{Chapter~\ref{chapter:research-01}: Representation learning using documental context } introduces the concept of learning multiple representations per word to capture lexical ambiguity in a language. 
We first investigate the influence of document topics on distinguishing different meanings of a word, then propose various models to integrate topical information in representation learning, and finally analyze the performance of these contextual representations and compare them to single representations. 
Our findings in in this chapter provide answers to \textbf{\ref{rq:topic}}.

\paragraph{Chapter~\ref{chapter:research-02}: Data augmentation for rare words} focuses on the impact of additional context in influencing translation quality of rare words. 
Notably, we use language models to substitute rare words in existing bilingual contexts. 
We augment the translation model with the newly generated data and as a result, improve both the generation frequency and the translation quality of rare words.
Our results in this chapter provide answers to \ref{rq:tda1} and \ref{rq:tda2}.

\paragraph{Chapter~\ref{chapter:research-03}: Data augmentation based on model failure} examines the influence of augmenting data with diverse context for difficult words on translation models. 
We first inspect the learning process of state-of-the-art translation models and identify where they are not confident in their predictions. 
After further analyzing the words that translation models have difficulties in learning, we introduce an augmentation approach to target these words.
We improve upon an existing data augmentation approach by devising new contexts for low-confidence words.
Our results in this chapter provide an answer to \ref{rq:bt1}~and \ref{rq:bt2}.

\paragraph{Chapter~\ref{chapter:research-04}: Analyzing idiomatic expressions} investigates translation errors prevalent in current models. 
First, we identify multiword expressions that are syntactically or semantically idiosyncratic and challenging to translate. 
Next, we create a parallel corpus consisting of sentence pairs with idiomatic expressions.
For this study, we introduce new error analysis measures to evaluate the translation quality of these expressions individually.
We provide empirical answers to \ref{rq:id1} and \ref{rq:id2} in this chapter.

\paragraph{Chapter~\ref{chapter:research-05}: Analyzing volatility} investigates the robustness of state-of-the-art translation models to variants in source sentences. 
We propose an effective technique to generate modifications in test sentences while avoiding the introduction of semantic or syntactic noise.
Investigating the translation outputs of different models on the modified test corpus reveals the extent of volatility that exists in translation models.
We perform an analysis of robustness of our models to answer \ref{rq:vol1} and \ref{rq:vol2}.

%Finally, we summarize findings from all research chapters in the concluding chapter:

\paragraph{Chapter~\ref{chapter:conclusions}: Conclusion} concludes this thesis by revisiting the research questions and their corresponding answers. 
We also reflect on future research directions and on what the community can learn from the findings in this thesis.




%% Describes the papers from which the chapters  originate
% !TEX root = thesis-main.tex

\section{Origins}
\label{section:introduction:origins}

The research presented in Chapters~\ref{chapter:research-01}-\ref{chapter:research-05} of this thesis is based on a number of peer-reviewed publications. 
Below, we indicate the origins of each chapter.

\paragraph{Chapter~\ref{chapter:research-01}} is based on Marzieh Fadaee and Arianna Bisazza and Christof Monz,
``Learning Topic-Sensitive Word Representations'',
\textit{In Proceedings of the 55th Annual Meeting of the Association for Computational Linguistics (ACL)},
%pages 441--447,
%Vancouver, Canada,
%July 2017.
\citep{fadaee-etal-2017-learning}.
Fadaee designed and carried out the experiments. All authors contributed to the discussion and text.

\paragraph{Chapter~\ref{chapter:research-02}} is based on Marzieh Fadaee and Arianna Bisazza and Christof Monz,
``Data Augmentation for Low-Resource Neural Machine Translation'',
\textit{In Proceedings of the 55th Annual Meeting of the Association for Computational Linguistics (ACL)},
%pages 567--573,
%Vancouver, Canada,
%July 2017.
 \citep{fadaee-bisazza-monz:2017:Short2}.
Fadaee designed the methods, performed the experiments and wrote most of the text.
Bisazza and Monz contributed to the discussion and editing.

\paragraph{Chapter~\ref{chapter:research-03}} is based on Marzieh Fadaee and Christof Monz,
``Back-Translation Sampling by Targeting Difficult Words in Neural Machine Translation'',
\textit{In Proceedings of the 2018 Conference on Empirical Methods in Natural Language Processing (EMNLP)},
%pages 436--446,
%Brussels, Belgium,
%October 2018.
\citep{fadaee-monz-2018-back}.
Fadaee designed the methods, performed the experiments and wrote most of the text.
Monz contributed to the discussion and editing.

\paragraph{Chapter~\ref{chapter:research-04}} is based on Marzieh Fadaee and Arianna Bisazza and Christof Monz,
``Examining the Tip of the Iceberg: A Data Set for Idiom Translation'',
\textit{In Proceedings of the Ninth International Conference on Language Resources and Evaluation (LREC)},
%pages 925--929,
%Miyazaki, Japan,
%May 2018.
\citep{L18-1148}.
Fadaee designed the methods, performed the experiments, and wrote the text.
Bisazza and Monz contributed to the discussion and editing.

\paragraph{Chapter~\ref{chapter:research-05}} is based on Marzieh Fadaee and Christof Monz,
``The Unreasonable Volatility of Neural Machine Translation'', 
\textit{In Proceedings of the 4th Workshop on Neural Generation and Translation (WNGT)},
%Seattle, Washington, USA.
%July 2020.
\citep{fadaee_new}.
Fadaee designed the methods, performed the experiments, and wrote the text.
Monz contributed to the discussion and editing.



